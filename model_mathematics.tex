\documentclass[12pt,letterpaper]{article}

%_______________________________________________________________

\usepackage{geometry}
 \geometry{
 a4paper,
 total={170mm,257mm},
 left=20mm,
 top=10mm,
 }
%Original packages - must be in this order
\usepackage{amsmath}
\usepackage{tcolorbox}
\usepackage{amssymb}
\usepackage{amsthm}
\usepackage{lastpage}
\usepackage{fancyhdr}
\usepackage{accents} 

%My packages
\usepackage{amsfonts}
\usepackage{dsfont}
\usepackage{graphicx}
\usepackage{mathrsfs}
\usepackage{mathtools}
\usepackage{multicol} % colors
\usepackage{relsize}
\usepackage{titlesec}
\usepackage{parskip} % paragraph structure 
\usepackage{biblatex} % bibliography 
\addbibresource{refs.bib} %Imports bibliography file


%_______________________________________________________________
% Page setup
%_______________________________________________________________

%   \titleformat{\section}[runin]{\fontsize{12}{15pt}\selectfont\bfseries}	{\thesection.}{0.5em}{} 
%   \titlespacing{\section}{0pc}{1.5ex plus .1ex minus .2ex}{1pc}

%\pagestyle{fancy}
\pagestyle{empty} % toggles page numbers
\setlength{\headheight}{42pt}
%\setlength{\parindent}{0pt} %set indent length
  
%_______________________________________________________________

\newcommand{\Z}{\mathbb{Z}}
\newcommand{\N}{\mathbb{N}}
\newcommand{\R}{\mathbb{R}}
\newcommand{\C}{\mathbb{C}}
\newcommand{\Q}{\mathbb{Q}}
\newcommand{\lrangle}[1]{\langle #1 \rangle}
\newcommand{\inv}{^{-1}}
\newcommand{\Lra}{\Leftrightarrow}
\newcommand{\ra}{\rightarrow}
\newcommand{\Ra}{\Rightarrow}
\newcommand{\La}{\Leftarrow}
\renewcommand{\l}{\left(}
\renewcommand{\r}{\right)}
%\newcommand{\qed}[0]{\hfill$\blacksquare$}
\renewcommand{\o}[1]{\overline{#1}}
\newcommand{\flag}{\textcolor{red}{*****}}
\newcommand{\ds}[1]{\begin{color}{blue}#1\end{color}}

%_______________________________________________________________
%_______________________________________________________________
%_______________________________________________________________
%_______________________________________________________________
%_______________________________________________________________

\begin{document}

\begin{center}
	{\Large{\bf Lexington Bikes Optimization Model}}
\end{center}

\section{Introduction} 

This document details the mathematical specifics of the optimization model used in this project. 

At the most basic level, this model should give an answer the question ``Where are the best places to install new bike lanes?''. 

In more detail, cycling infrastructure can be classified as low-stress or high-stress, see \cite{mti} and \cite{pfb}. 
We model a city with network flow: people traveling from census blocks, through the bike network, to destinations. 
We assume people want to bike exclusively on low-stress infrastructure, but must accept use of high-stress segments due to the car-centric design of many North American cities. 
In this setup, we imagine a city has a budget to add new bicycle infrastructure, such as protected lanes, to existing roadway to convert high-stress segments to low-stress. 
For each possible arrangement of newly low-stress infrastructure, we calculate total distance traveled and keep track of the configuration which resulted in the lowest total trip distance. 

\section{Condensed Model}

We must solve the following optimization problem: 
\begin{align}
    \text{minimize} \hspace{0.5cm}& w := \sum_{(i,j)\in A} d_{ij}x_{ij}+ \sum_{(i,j)\in A} (f-1)d_{ij}z_{ij},\label{eq:objective}\\
    \text{subject to} \hspace{0.5cm}& \sum_{(n,j)\in A} x_{nj} - \sum_{(i,n)\in A} x_{in} = b_n, \quad \forall n \in N; \label{eq:netflow}\\
    & z_{ij}\geq x_{ij}-Py_{ij} , \quad\forall (i,j)\in H; \label{eq:stress-logic}\\
    & y_{ij} \leq y_{ji}, \quad\forall (i,j)\in H_2; \label{bi-directional upgrades} \\
    & \sum_{(i,j)\in H_1} u_{ij}y_{ij} + \sum_{(i,j)\in H_2} v_{ij}y_{ij}\leq B; \label{eq:budget} \\
    & x_{ij} \in \Z_{\geq 0}, \quad\forall (i,j) \in A \label{eq:integral_x} \\
    & y_{ij} \in \{0,1\}, \quad \forall (i,j) \in H \label{eq:binary_y} \\
    & z_{ij} \in \{0,1\}, \quad \forall (i,j) \in H \label{eq:binary_z}
\end{align}

Constraint \ref{eq:netflow} ensures that each node has the correct in- and out-flow. 
Constraint \ref{eq:stress-logic} adds a penalty to the objective function in the case that a non-upgraded high stress arc is used, and relies on the assumption that no trip will make a loop, and hence no arc will have more than $P$ trips across it. 
Constraint \ref{eq:budget} ensures that the money spent on upgrades will not exceed the total budget. 
Constraint \ref{eq:integral_x} ensures that the number of trips is a nonnegative integer. 
Constraint \ref{eq:binary_y} ensures that the upgrade status of high stress arcs is binary. 



\section{Explanations}


\subsection{Sets} 

$N=$ set of nodes.

$O = $ set of origins (census blocks) $\subset N$. Define a dummy origin node $o^*\in O$ that doesn't represent a census block, but rather is used to balance supply and demand between origin and destination nodes. 

$D= $ set of destinations (amenities of a single type) $\subset N$.

$A = $ set of arcs, of the form $(i,j)$ for $i,j\in N$.

$L= $ set of low stress arcs $\subset A$.

$H= $ set of high stress arcs $\subset A$.

Note that $H$ and $L$ form a partition of $A$. 

\subsection{Parameters}

$d_{ij}=$ distance of arc $(i,j)$, for $(i,j)\in A$. 

$p_n=$ population at origin node $n$, for $n\in O$. 

$b_n=$ net flow at node $n$ $=$ (supply at $n$) $-$ (demand at $n$), for $n\in N$. In particular, define
\begin{align*}
    b_n &= \left\{\begin{array}{ll}
        -P, & n=o^* \\
        p_n, & n\in O  \\
        %0, & n\in D \\
        0 & \mbox{else}
    \end{array}\right.  .
\end{align*}
where $o^*$ is a dummy node. 


$P=\sum_{n\in O} p_n$ total population. (Note: This is redundant to adding up the supply at each origin node.)

$u_{ij}=$ cost to upgrade arc $(i,j)\in H$ to become low stress, for $(i,j)$.

$B=$ total budget.

$f=$ the factor by which a low stress path would need to exceed a high stress route in order for someone to choose the shorter high stress route. Equivalently, we can say someone would be willing $f$ times further in order to stay on a low stress path. 

\subsection{Decision Variables}

$x_{ij}=$ number of trips on arc $(i,j)$, for $(i,j)\in A$. 

$y_{ij}=\left\{\begin{array}{ll}
    0, & (i,j) \mbox{ not upgraded to low stress} \\
    1, & (i,j) \mbox{ upgraded to low stress}
\end{array}\right. $ , for $(i,j)\in H$. 

$z_{ij}=\left\{\begin{array}{ll}
    0, & (i,j) \mbox{ not upgraded to low stress} \\
    x_{ij}, & (i,j) \mbox{ upgraded to low stress}
\end{array}\right. $ , for $(i,j)\in H$. 

\subsection{Objective Function}

Goal: Minimize the total trip distance, while including the penalty for using high stress arcs. 
\[ \sum_{(i,j)\in A} d_{ij}x_{ij} + \sum_{(i,j)\in H} (f-1)d_{ij}z_{ij} .\]

\subsection{Constraints}

Each arc $(i,j)\in A$ has a whole, nonnegative number of trips:
\[ x_{ij}\in\Z_{\geq0} , \quad \forall (i,j)\in A. \]

The flow must be balanced at each node $n\in N$. That is, the number of outgoing trips $-$ the number of incoming trips must equal the net flow:
\[ \sum_{(n,j)\in A} x_{nj} - \sum_{(i,n)\in A} x_{in} = b_n , \quad \forall n\in N.\]

Trips can use low stress arcs, or high stress arc that have been upgraded without penalty. 
If a trip uses a high stress arc, it will incur a penalty in the objective function. 
Assuming trips don't make loops, we get that $x_{ij}\leq P$ for all $(i,j)\in A$. 
Using this, we can model the above constraint with the inequality with a `big $M$' constraint, where the total population $P$ serves as our $M$. 
\[z_{ij}\geq x_{ij}-Py_{ij} , \quad \forall (i,j)\in H .\]

Stay within budget:
\[ \sum_{(i,j)\in H} u_{ij}y_{ij} \leq B. \]

\section{Lexington}

For simplicity, we begin with parks as the only destination type. 
Additionally, we limited our data to the city's Urban Service Area.
Further, the bike network data from \cite{pfb} has a handful of parking lots, golf course, etc. that are not connected to the larger network, and we removed these. 

\printbibliography

\end{document}
